\documentclass[11pt,a4paper]{article}
\title{Change cursor face of Linux Console}
\author{Sugizaki Yukimasa (Namiki Secondary School, Ibaraki Prefecture)}
\date{\today}

\begin{document}
\maketitle
\abstract{
In Linux console, shapes, colours and blinking can be changed.
This paper explains how to do this.

\tableofcontents

\section{By using Escape squences}
Define escape charactor as \texttt{ESC}.

The format is below. \\
\verb|ESC[?A;B;Cc|
(\texttt{A}, \texttt{B} and \texttt{C} are 8-bit decimal numbers.)

\subsection{Parameter \texttt{A}}
Parameter \texttt{A} changes cursor shapes and states.
Table\ref{tab:param-A-effects} shows effects of parameter \texttt{A}.
\begin{table}[htb]
\begin{center}
\begin{tabular}{|c||cl|}
\hline
number & effect \\
\hline\hline
0 & default \\
1 & underline \\
$\cdots$	\\
8 & full block \\
\hline
$number+16$ & apply the software cursor \\
\end{tabular}
\label{tab:param-A-effects}
\caption{Effects of parameter \texttt{A}}
\end{center}
\end{table}
Cursor and characters colour can be changed by the parameter \texttt{A} and \texttt{B}
if the software cursor applied.

\subsection{Parameter \texttt{B}}
Parameter \texttt{B} changes the colour and brightness of cursor and charcters.
Table \ref{tab
\subsection{Parameter \texttt{C}}
\end{document}
